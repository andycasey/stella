%% This file is part of the Stella Project 
%% Copyright 2016 Andrew R. Casey.  All rights reserved.

\documentclass[preprint]{aastex}

\usepackage{amsmath}
\usepackage{bm}
\usepackage{mathtools} % Provides amsmath and dcases so that \fracs are not compressed in \cases environments

\IfFileExists{vc.tex}{\input{vc.tex}}{\newcommand{\githash}{UNKNOWN}\newcommand{\giturl}{UNKNOWN}}

\newcommand{\acronym}[1]{{\small{#1}}}

\newcommand{\project}[1]{\textsl{#1}}
\newcommand{\parsec}{\acronym{PARSEC}}

\newcommand{\teff}{T_{\mathrm{eff}}}
\newcommand{\teffSolar}{T_{\mathrm{eff},\odot}}
\newcommand{\logg}{\log g}
\newcommand{\feh}{[\mathrm{Fe/H}]}

% Common terms.
\newcommand{\parallax}{\varpi}
\newcommand{\mass}{\mathcal{M}}
\newcommand{\massSolar}{\mass_\odot}
\newcommand{\numax}{\nu_\mathrm{max}}
\newcommand{\numaxSolar}{\nu_{\mathrm{max},\odot}}
\newcommand{\deltanu}{\Delta\nu}
\newcommand{\deltanuSolar}{\deltanu_\odot}
\newcommand{\age}{\tau}

\newcommand{\given}{|}
\newcommand{\data}{\mathcal{D}}

\newcommand{\uniform}[2]{\mathcal{U}\left(#1,#2\right)}

\begin{document}
\title{Stella}

\author{
  Andrew~R.~Casey\altaffilmark{1}
}


\begin{abstract}
  Some stuff
\end{abstract}


\keywords{stars: fundamental parameters --- stars: abundances
% Clear rest of page
%\clearpage}
}

\section{Introduction} 
\label{sec:introduction}


The asteroseismic scaling relation for mass is

\begin{eqnarray}\label{eq:mass-scaling-relation}
    \frac{\mass}{\massSolar} \approx \left(\frac{\numax}{\numaxSolar}\right)^{3}\left(\frac{\deltanu}{\deltanuSolar}\right)^{-4}\left(\frac{\teff}{\teffSolar}\right)^{3/2}  \quad ,
\end{eqnarray}

\noindent{}but just how wrong is this? There are good theoretical reasons to expect that the power spectral density properties $\numax$ and $\deltanu$ are dependent on the stellar chemical composition (i.e., $\feh$), yet there is no metallicity dependence given in \ref{eq:mass-scaling-relation}.  Moreover, if the scaling relations are employed for extremely metal-poor stars, then the inferred masses are $\gtrsim2M_\odot$, implying that those extremely metal-poor stars must be very young. Naturally, this is inconsistent with the general consensus that extremely metal-poor stars are old, ancient objects.

So, how wrong are the scaling relations? Is there a metallicity dependence that we can incorporate into Equation \ref{eq:mass-scaling-relation} to bring the inferred masses of extremely metal-poor stars consistent with expectations from stellar evolution? 

\section{Model}
\label{sec:model}


Our model has the following parameters (summarised as $\theta$): Current mass $\mass_{now}$, distance $d$, initial mass $\mass_{init}$, stellar age $\age$, and photospheric metallicity $\feh$:
\begin{eqnarray}\label{eq:model-parameters}
    \theta = (\mass_{now}, d, \mass_{init}, \tau, \feh)
\end{eqnarray}

We seek to generate the data from these model parameters. Specifically from a given initial mass $\mass_{init}$, metallicity $\feh$ and age $\tau$ we interpolate a \parsec\ isochrone.  We interpolate along that isochrone to the proposed current mass $\mass_{now}$, and yield the corresponding absolute $J$- and $H$-band magnitudes $M_J$ and $M_H$, respectively.  Given the distance $d$ and by adopting extinction from the \citet{Schlegel} maps (\textbf{someone must have carefully done dust in the kepler field -- use that}), the apparent magnitude in $J$ (or $H$) can be derived: 
\begin{eqnarray}
    J = M_J + 5\log_{10}d - 5 + A_J \quad .
\end{eqnarray}

The adopted extinction in each band is given for all sources in Table \ref{tab:stellar-properties}.


Thereby we can generate the observed parallax $\parallax$ ($1/d$), apparent $J$-band magnitude, and apparent $J - H$ color, which we will collectively refer to as the data $\data = (\parallax, J, J-H)$. 

\textbf{Proper use of priors is required here, and we will use bayes}


We adopt the log likelihood function:
\begin{eqnarray}\label{eq:likelihood-function}
    \log{p\left(\data\given\theta\right)} & = & -\frac{1}{2}\left[\frac{(J-J_{model})^2}{\sigma_{J}^2} + \frac{1}{\sigma_\parallax^2}\left(\parallax - \frac{1}{d}\right)^2 + \frac{\left([J-H] - [J-H]_{model}\right)^2}{\sigma_J^2 + \sigma_H^2}\right] \quad . 
\end{eqnarray}


We adopt a uniform prior on the current mass $\mass_{now} = \uniform{0.15}{\mass_{init}}$, allowing for mass loss throughout the stellar lifetime down to the minimum mass in any \parsec\ isochrone.  We adopt a prior on the distance $d$ that assumes an exponentially decreasing space density with increasing distance,
\begin{eqnarray}
    p\left(d\right) = \begin{dcases*}
        \frac{d^2}{2L^3}\exp{\left(-\frac{d}{L}\right)} & if $d > 0$,\\
        0 & otherwise.
    \end{dcases*}
\end{eqnarray}

\noindent{}where we adopt a length scale $L = 1.35$~kpc.

\textbf{Joint priors on initial mass, age, and metallicity: Use the spectroscopic metallicity as a prior for each star, then otherwise use the Binney joint priors?}




\end{document}
